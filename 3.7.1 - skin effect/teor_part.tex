\documentclass{article}
\usepackage[utf8]{inputenc}

\documentclass[12pt, a4paper]{article}
\usepackage[utf8]{inputenc}
\usepackage{amsmath,amsthm,amssymb}
\usepackage[T2A]{fontenc}
\usepackage[english,russian]{babel}
\usepackage{mathtext}
\usepackage{wrapfig}
\usepackage{graphicx}
\usepackage{mathtext}
\usepackage{amsmath}
\usepackage{siunitx}
\usepackage{multirow}
\usepackage{rotating}
\usepackage{float}
\usepackage{booktabs}
\usepackage{color, colortbl}

\begin{document}
    	\begin{center}
		\large 	Московский физико-технический университет \\
		Факультет аэрофизики и космических исследований \\
		\vspace{0.2cm}
		
		\vspace{4.5cm}
		Лабораторная работа № 3.7.1 \\ \vspace{0.2cm}
		\large (Общая физика: электричество и магнетизм) \\ \vspace{0.2cm}
		\LARGE \textbf{Скин-эффект в полом цилиндре}
	\end{center}
	\vspace{2.3cm} \large
	
	\begin{center}
		Работу выполнил: \\
		Кривенко Павел,
		группа Б03-101
		\vspace{10mm}
		
	
		
		
	\end{center}
	
	\begin{center} \vspace{45mm}
		г. Долгопрудный \\
		 2022 год
	\end{center}
\end{titlepage}

\paragraph*{Цель работы:} исследование проникновения переменного магнитного поля в медный полый цилиндр.

\paragraph*{Оборудование:} генератор звуковой частоты, соленоид, намотанный на полый цилиндрический каркас из диэлектрика, медный экран в виде трубки, измерительня катушка, амперметр, вольтметр, осциллограф.

\section{Теоретическое введение}
В работе изучается скин-эффект в длинном тонкостенном медном цилиндре,помещенном внутрь соленоида.Данная задача требует решение уравнений скин-эффекта(уравнений диффузии поля) в стенке цилиндра и квазистационарных уравнений поля в его полости.\newline Пусть цилиндр достаточно длинный, так что в нем можно пренебречь краевыми эффектами. В этом приближении магнитное поле \textbf{H} всюду направлено по оси системы (ось z), а вихревое электрическое поле \textbf{E} будет всбду перпендикулярно радиусу, то есть линии поля образуют соосные окружности (рис. 1). Все величины будем считать колеблющимися по гармоническому закону с некоторой частотой $\omega$ задаваемой частотой колебания тока в соленоиде. Тогда для ненулевых компонент поля можно записать
\newline $ H_z = H(r) e^{i\omega t} $, $E_\phi = E(r) e^{i\omega t}$, где H(r) и E(r) - комплексные амплитуды колебаний соответствующих полей, зависящие только от расстояния r до оси системы. (Функции H(r) и E(r) - непрерывны на всей исследуемой области). 
\newline Пусть длинный полый цилиндр имеет радиус a и толщину стенки h \ll a. Данной условие позволяет ограничиться одномерным приближением. 
\newline Поскольку внутри цилиндра ток отсутсвует, магнитное поле там является однородным: $H_z(r, t) = H1e^{i\omega t} $, где $H_1 = const$ - амплитуда поля на внутренней поверхности цилиндра. Для нахождения вихревого электрического поля воспользуемся законом элеткромагнитной индукции в интегральной форме:
\newline $E_\phi 2 \pi r = -\mu_0 \pi r^2 \dfrac{dH_z}{dt}$, то есть $E(r) = -\dfrac{1}{2} \mu_0 r i \omega H1$.
\newline Отсюда получим связь амплитуд колебаний электрического и магнитного полей на внутренней (r = a) границе цилиндра:
\centering $E_1 = -\dfrac{1}{2} i \omega a \mu_0 H_1$. (1)
\newline соотношение (1) используется как дополнительное граничное условие для задачи о рапсределении поля внутри стенки. 
\newline
\\ Поле внутри тонкой стенки цилиндра описыавется уравнением скин-эффекта:
$\dfrac{\partial^2 E_y}{\partial x^2} = \sigma \mu \mu_0 \dfrac{\partial E_y}{\partial t}$. 
\newline Поместим начало отсчета на внешнюю поверхность цилиндра и направим ось x к оси системы, и запишем дифференциальное уравнение для комплексной амплитуды магнитного поля:
\centering $\dfrac{d^2 H}{dx^2} = i\omega \sigma \mu_0 H$ (2) 
\newline (Для медного цилиндра можно положить $\mu = 1$)
\newline Граничные условия: $H(0) = H_0$, $H(h) = H_1$. (3)
\newline Здесь $H_0$ - амплитуда колебаний магнитного поля на внешней границе цилиндра. Ее знаечение определяется только током в обмотке соленоида, и совпадает с полем внутри соленоида в отсутсвие цилиндра. Величина $H_1$ - это амплитуда колебаний однородного поля внутри цилиндра. Связь полей $H_0$ и $H_1$ выражена соотношением (1). 
\newline
\newline Решение (2) ищем в виде: 
\centering $H(x) = A e^{\alpha x} + B e^{-\alpha x}$,  (4)
\newline где А и В - определяемые из граничных условий константы, 
\centering $\alpha = \sqrt{i \omega \sigma \mu_0} = \dfrac{1 + i}{\delta} = \dfrac{\sqrt{2}}{\delta} e^{i \pi/4}$,  (5)
\newline 
\newline где $\delta$ - глубина скин-слоя
\newline Первое условие (3) дает $A + B = H_0$, что позволяет исключить А из (4):
\centering $H(x) = H_0 e^{-\alpha x} + 2B\sh{\alpha x}$
\newline Выразим электрическое поле из закона Ампера. В одномерном случае:
\centering $E(x) = \dfrac{1}{\sigma} \dfrac{dH}{dx} = \dfrac{\alpha}{\sigma} (-H_0 e^{-\alpha x} + 2B\ch{\alpha x})$
\newline Далее положим $x = h$, воспользуемся условием (1), и, исключив константу В, получим после преобразований связь между $H_0$ и $H_1$:
\newline
\centering $H_1 = \dfrac{H_0}{\ch{\alpha h} + \dfrac{1}{2} \alpha a \sh{\alpha h}}$ (6)
\newline
\newline Рассмотрим предельные случаи (6).
\newline 1. При малых частотах толщина скин-слоя превосходит толщину цилиндра $\delta \gg h $. Тогда $|\alpha h| \ll 1$, поэтому $\ch{\alpha h} \approx 1$, $\sh{\alpha h} \approx \alpha h и $
\newline
\centering $H_1 \approx \dfrac{H_0}{1 + i\dfrac{ah}{\delta^2}}$  (7)
\newline
\newline Отношение модулей амплитуд здесь будет равно
\newline
\centering $\dfrac{|H_1|}{|H_0|} = \dfrac{1}{\sqrt{1 + (\dfrac{ah}{\delta^2})^2}} = \dfrac{1}{\sqrt{1 + \dfrac{1}{4}(ah\sigma\mu_0\omega)^2}}$.   (8)
\newline
\newline При этом колебания $H_1$ отстают о фазе от $H_0$ на угол $\psi$, определяемый равенством $tg\psi = \dfrac{ah}{\delta^2}$.
\newline 2. При достаточно больших частотах толщина скин-слоя станет меньше толщины стенки: $\delta \ll h$ и выражение (6) с учетом (5) переходит в 
\newline
\centering $\dfrac{H_1}{H_0} = \dfrac{4}{\alpha a} e^{-\alpha h} = \dfrac{2\sqrt{2} \delta}{a} e^{-\dfrac{h}{\delta}} e^{-i(\dfrac{\pi}{4} + \dfrac{h}{\delta})}$  (9)
\newline
\newline Как видно из формулы (9), в этом пределе поле внутри цилиндра по модулю в $\dfrac{2\sqrt{2} \delta}{a} e^{-\dfrac{h}{\delta}}$ раз меньше, чем снаружи, и, кроме того, запаздывает по фазе на 
\newline
\centering $\psi = \dfrac{\pi}{4} + \dfrac{h}{\delta} = \dfrac{\pi}{4} + h \sqrt{\dfrac{\omega \sigma \mu_0}{2}}$.      (10)










\end{document}
